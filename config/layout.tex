% The memoir class used in this template contains the ability to set the stock
% paper size and the trimmed size independently. It also has the ability to show
% trim lines showing where stock paper should be trimmed to get the final book
% size. This can all be a bit confusing so please see the memoir class
% documentation for more information.

% By default, the paper size is a4paper which is 29.7cm × 21cm. To change this,
% simply change "a4paper" in the \documentclass[a4paper,...]{memoir} command in
% thesis.tex to another size such as "letterpaper". By default, the trimmed size
% is 24cm x 17cm and trim lines are shown. To remove trim lines, simply remove
% "showtrims" from the \documentclass[showtrims,...]{memoir} command in
% thesis.tex. The size of the trimmed content is set with the
% \settrimmedsize{}{} command below. If you wish to remove trims and set the
% content to fit the paper size (i.e. no trimming at all), all you have to do is
% remove "showtrims" as above and comment out the \settrimmedsize{}{} command
% below.

% Uncomment to manually set the stock size and override the setting in \documentclass.
%\setstocksize{24cm}{17cm}
% Change the trimmed area size or comment out this line entirely to fit the
% content to the paper size without trimming.
\settrimmedsize{24cm}{17cm}{*}
% The first bracket specifies the spine margin, the second the edge margin and
% the third the ratio of the spine to the edge. Only one or two values are
% required and the remaining one(s) can be a star (*) to specify it is not
% needed.
\setlrmarginsandblock{22mm}{*}{0.9}
% The first bracket specifies the upper margin, the second the lower margin and
% the third the ratio of the upper to the lower. Only one or two values are
% required and the remaining one(s) can be a star (*) to specify it is not
% needed.
\setulmarginsandblock{26mm}{20mm}{*}


% The size of marginal notes, the three values in curly brackets are
% \marginparsep, \marginparwidth and \marginparpush.
\setmarginnotes{17pt}{51pt}{\onelineskip}
% Sets the space available for the header and footer
\setheadfoot{\onelineskip}{2\onelineskip}
% Sets the spacing above and below the header
\setheaderspaces{*}{2\onelineskip}{*}


% Sets the spacing above the trimmed area, i.e. moved the trimmed area down the
% page if positive.
\setlength{\trimtop}{0pt}


% Comment the two lines below to reverse the position of the trimmed content on
% the stock paper, i.e. odd pages will have content on the right side instead of
% the left and even pages will have content on the left side instead of the
% right.
\setlength{\trimedge}{\stockwidth}
\addtolength{\trimedge}{-\paperwidth}

% To bring content to center.
\addtolength{\trimtop}{2.85cm}
% To bring content to center.
\addtolength{\trimedge}{-2cm}

% Display other style of trim marks.
\quarkmarks

% Put jobname in left top trim mark.
\renewcommand*{\tmarktl}{\registrationColour{%
  \begin{picture}(0,0)
    \setlength{\unitlength}{1bp}\thicklines
    \put(-36,0){\line(1,0){24}}
    \put(0,12){\line(0,1){24}}
    \put(3,27){\normalfont\ttfamily\fontsize{8bp}{10bp}\selectfont\jobname\ \
      \today\ \ \printtime\ \ Sheet \thesheetsequence}
  \end{picture}}}


% Makes sure your specifications are correct and implements them in the document.
\checkandfixthelayout


% Text breakpoint setup.
\raggedbottom
\hyphenpenalty=5000
\tolerance=1000